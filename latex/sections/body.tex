\section{Introduction}
\begin{quotation}
    \emph{"And if you gaze long into an abyss, the abyss also gazes into you."}
    --- Friedrich Nietzsche \emph{Beyond Good and Evil}, Aphorism 146
\end{quotation}

\noindent
A pinwheel tiling is a special form of tiling that is based on the an 1, 2, right triangle - where each triangle can be divided into 5 isometric copies of itself. Furthermore, this tiling is aperiodic, allowing one to create mesmerizing patterns that bear only a hint of crystalline order.

\subsection{Rationale}
There's something beautiful about the nature of infinities, especially ones of the fractal and recursive kind. As a programmer, I deal frequently with recursion -- a subject that is intimately related to the nature of convergence, limits, and calculus in mathematics. The pinwheel tiling is a form of tiling that is recursive - for with each pinwheel triangle, you can divide it into 5 isometric parts.

I had first heard of this property when doing texture-mapping work for video games, for such a recursion is useful when making textures that scale well. This is why I decided to pursue this topic for my investigation.

\subsection{Aims of this investigation}
For this investigation, I aim to explore two major aspects of the pinwheel triangle. The first aspect that I would wish to explore is the properties of pinwheel triangles, and how they are affected by this recursive nature. Specifically, I would wish to discover the \emph{area} of each pinwheel triangle at an arbitrary stage of recursion, as well as essential metrics such as the angles involved.

This would allow me to set up the prerequisite knowledge for the core of my exploration, which is the pinwheel spiral. A pinwheel spiral can be constructed out of the vertices of an recursively constructed pinwheel triangle, and I would wish to find out it's properties such as it's length, using the tools of geometric sequences and limits, as well as graphical methods.

To this end, I will be dividing my investigation into three parts. The first section will be a construction of the pinwheel triangle, the second section on the geometrical properties of the pinwheel triangle, and the third section on the geometric, and convergent properties of the pinwheel spiral.

\subsection{Terminology}
Except when otherwise noted, I will strive to use standard terminology from the IB mathematics guidebook.

\newpage
\section{Construction of the pinwheel triangle}
\begin{enumerate}
    \item Draw $\triangle A_{0}B_{0}C_{0}$ where:
    \begin{equation}
        \begin{aligned}
            \angle A_{0}B_{0}C_{0} &= \frac{\pi}{2} \\
            B_{0}C_{0} &= 2A_{0}B_{0} \\
            A_{0}C_{0} &= \sqrt{(A_{0}B_{0})^2 + (B_{0}C_{0})^2} \\
        \end{aligned}
    \end{equation}
    \begin{figure}[H]
        % This figure has to be inlined rather than using input, due to the enumerate environment used.
        \centering
        \begin{tikzpicture}[scale=5]
            \coordinate [label=above: $A_{0}$] (A_0) at (0,1);
            \coordinate [label=below: $B_{0}$] (B_0) at (0,0);
            \coordinate [label=right: $C_{0}$] (C_0) at (2,0);
            \draw (A_0) -- (B_0) -- (C_0) -- (A_0);
        \end{tikzpicture}
        \label{right-triangle}
        \caption{Construction of the $\protect\triangle A_{0}B_{0}C_{0}$}
    \end{figure}

    \item Draw line $B_{0}A_{1}$, where $B_{0}A_{1} \perp A_{0}C_{0}$.
    \item Draw point $C_{1}$ at the midpoint of $B_{0}C_{0}$.
    \item Draw line $C_{1}H_{0}$, where $C_{0}H_{0} \perp A_{0}C_{0}$.
    \item Draw line $B_{1}C_{1}$, where $B_{1}C_{1} \perp B_{0}A_{1}$.
    \item Draw line $A_{1}C_{1}$.
\end{enumerate}

\begin{figure}[H]
    \centering
    \begin{tikzpicture}[scale=5]
        \coordinate [label=above: $A_{0}$] (A_0) at (0,1);
        \coordinate [label=below: $B_{0}$] (B_0) at (0,0);
        \coordinate [label=right: $C_{0}$] (C_0) at (2,0);
        \coordinate [label=above: $A_{1}$] (A_1) at ($(A_0)!(B_0)!(C_0)$);
        \coordinate [label=below: $C_{1}$] (C_1) at ($(B_0)! 0.5 !(C_0)$);
        \coordinate [label=above: $H_{0}$] (H_0) at ($(A_0)!(C_1)!(C_0)$);
        \coordinate [label=left: $B_{1}$]  (B_1) at ($(A_1)!(C_1)!(B_0)$);

        \draw (A_0) -- (B_0) -- (C_0) -- (A_0);
        \draw (A_1) -- (B_0);
        \draw (C_1) -- (H_0);
        \draw (B_1) -- (C_1);
        \draw (A_1) -- (C_1);
    \end{tikzpicture}
    \label{pinwheel-triangle}
    \caption{Simple pinwheel triangle}
\end{figure}


\subsection{Recursive pinwheel tiling}
Repeat steps 2 through 6, operating on each subsequent $\triangle A_{i}B_{i}C_{i}$.
\begin{figure}[H]
    \centering
    \begin{tikzpicture}[scale=5]
        \coordinate [label=above: $A_{0}$] (A_0) at (0,1);
        \coordinate [label=below: $B_{0}$] (B_0) at (0,0);
        \coordinate [label=right: $C_{0}$] (C_0) at (2,0);
        \draw (A_0) -- (B_0) -- (C_0) -- (A_0);

        \coordinate [label=above: $A_{1}$]
        (A_1) at ($(A_0)!(B_0)!(C_0)$);
        \coordinate [label=below: $C_{1}$]
        (C_1) at ($(B_0)!0.5!(C_0)$);
        \coordinate [label=above: $H_{0}$]
        (H_0) at ($(A_0)!(C_1)!(C_0)$);
        \coordinate [label=left: $B_{1}$]
        (B_1) at ($(A_1)!(C_1)!(B_0)$);
        \coordinate [label=above: $A_{2}$]
        (A_2) at ($(A_1)!(B_1)!(C_1)$);
        \coordinate [label=below: $C_{2}$]
        (C_2) at ($(B_1)!0.5!(C_1)$);
        \coordinate [label=above: $H_{1}$]
        (H_1) at ($(A_1)!(C_2)!(C_1)$);
        \coordinate [label=left: $B_{2}$]
        (B_2) at ($(A_2)!(C_2)!(B_1)$);

        \coordinate (A_3) at ($(A_2)!(B_2)!(C_2)$);
        \coordinate (C_3) at ($(B_2)! 0.5 !(C_2)$);
        \coordinate (H_2) at ($(A_2)!(C_3)!(C_2)$);
        \coordinate (B_3) at ($(A_3)!(C_3)!(B_2)$);
        \coordinate (A_4) at ($(A_3)!(B_3)!(C_3)$);
        \coordinate (C_4) at ($(B_3)! 0.5 !(C_3)$);
        \coordinate (H_3) at ($(A_3)!(C_4)!(C_3)$);
        \coordinate (B_4) at ($(A_4)!(C_4)!(B_3)$);
        \coordinate (A_5) at ($(A_4)!(B_4)!(C_4)$);
        \coordinate (C_5) at ($(B_4)! 0.5 !(C_4)$);
        \coordinate (H_4) at ($(A_4)!(C_5)!(C_4)$);
        \coordinate (B_5) at ($(A_5)!(C_5)!(B_4)$);
        \coordinate (A_6) at ($(A_5)!(B_5)!(C_5)$);
        \coordinate (C_6) at ($(B_5)! 0.5 !(C_5)$);
        \coordinate (H_5) at ($(A_5)!(C_6)!(C_5)$);
        \coordinate (B_6) at ($(A_6)!(C_6)!(B_5)$);
        \coordinate (A_7) at ($(A_6)!(B_6)!(C_6)$);
        \coordinate (C_7) at ($(B_6)! 0.5 !(C_6)$);
        \coordinate (H_6) at ($(A_6)!(C_7)!(C_6)$);
        \coordinate (B_7) at ($(A_7)!(C_7)!(B_6)$);
        \coordinate (A_8) at ($(A_7)!(B_7)!(C_7)$);
        \coordinate (C_8) at ($(B_7)! 0.5 !(C_7)$);
        \coordinate (H_7) at ($(A_7)!(C_8)!(C_7)$);
        \coordinate (B_8) at ($(A_8)!(C_8)!(B_7)$);
        \coordinate (A_9) at ($(A_8)!(B_8)!(C_8)$);
        \coordinate (C_9) at ($(B_8)! 0.5 !(C_8)$);
        \coordinate (H_8) at ($(A_8)!(C_9)!(C_8)$);
        \coordinate (B_9) at ($(A_9)!(C_9)!(B_8)$);

        \draw (A_1) -- (B_0);
        \draw (C_1) -- (H_0);
        \draw (B_1) -- (C_1);
        \draw (A_1) -- (C_1);
        \draw (A_2) -- (B_1);
        \draw (C_2) -- (H_1);
        \draw (B_2) -- (C_2);
        \draw (A_2) -- (C_2);
        \draw (A_3) -- (B_2);
        \draw (C_3) -- (H_2);
        \draw (B_3) -- (C_3);
        \draw (A_3) -- (C_3);
        \draw (A_4) -- (B_3);
        \draw (C_4) -- (H_3);
        \draw (B_4) -- (C_4);
        \draw (A_4) -- (C_4);
        \draw (A_5) -- (B_4);
        \draw (C_5) -- (H_4);
        \draw (B_5) -- (C_5);
        \draw (A_5) -- (C_5);
        \draw (A_6) -- (B_5);
        \draw (C_6) -- (H_5);
        \draw (B_6) -- (C_6);
        \draw (A_6) -- (C_6);
        \draw (A_7) -- (B_6);
        \draw (C_7) -- (H_6);
        \draw (B_7) -- (C_7);
        \draw (A_7) -- (C_7);
        \draw (A_8) -- (B_7);
        \draw (C_8) -- (H_7);
        \draw (B_8) -- (C_8);
        \draw (A_8) -- (C_8);
        \draw (A_9) -- (B_8);
        \draw (C_9) -- (H_8);
        \draw (B_9) -- (C_9);
        \draw (A_9) -- (C_9);

    \end{tikzpicture}
    \label{pinwheel-triangle-infinite}
    \caption{Recursive pinwheel triangle}
\end{figure}


\newpage
\begin{figure}[H]
    \centering
    \begin{tikzpicture}[rotate=90,line join=round]
      \begin{scope}[scale=8]
        \coordinate [label=left: $A_{0}$] (A-0) at (0,1);
        \coordinate [label=right: $B_{0}$] (B-0) at (0,0);
        \coordinate [label=above: $C_{0}$] (C-0) at (2,0);
      \end{scope}
      \draw (A-0) -- (B-0) -- (C-0) -- (A-0);

      \foreach \lev in {1,...,12}{
        \pgfmathsetmacro{\plev}{int(\lev-1)}
        \coordinate (A-\lev) at ($(A-\plev)!(B-\plev)!(C-\plev)$);
        \coordinate (C-\lev) at ($(B-\plev)!.5!(C-\plev)$);
        \coordinate (H-\plev) at ($(A-\plev)!(C-\lev)!(C-\plev)$);
        \coordinate (B-\lev) at ($(A-\lev)!(C-\lev)!(B-\plev)$);
        \draw[thin] (A-\lev) -- (B-\plev);
        \draw[thin] (C-\lev) -- (H-\plev);
        \draw[thin] (A-\lev) -- (B-\lev) -- (C-\lev) -- cycle;
      }
    \end{tikzpicture}
    \label{pinwheel-triangle-infinite-large}
    \caption{Large, detailed view of the recursive pinwheel triangle}
\end{figure}



\section{Properties of the recursive pinwheel tiling}
\subsection{Angles}
For each angle in $\triangle A_{i}B_{i}C_{i}$, where the vertices are of the same level $n$ (e.g. $\angle A_{6}B_{6}C_{6}$). This property is apparent becasue each subsequent triangle is similar, therefore all angles are equal.
\begin{equation}
    \begin{aligned}
        \angle A_{i}B_{i}C_{i} &= \frac{\pi}{2} \\
        \angle B_{i}C_{i}A_{i} &= \tan^{-1}\frac{1}{2} \\
        \angle B_{i}A_{i}C_{i} &= \tan^{-1}2
    \end{aligned}
\end{equation}

\subsection{Area and dilation factor}
Let length $A_{0}B_{0} = x$. Define $\big<\triangle A_{i}B_{i}C_{i}\big>$ as the notation for the area of $\triangle A_{i}B_{i}C_{i}$. Therefore, $\big<\triangle A_{0}B_{0}C_{0}\big> = x^2$. Because $\triangle A_{0}B_{0}C_{0}$ contains five isometric subtriangles of the order $\triangle A_{1}B_{1}C_{1}$, the area of each $n = 1$ triangle equals $\frac{x^2}{5}$. Recall that length $A_{1}C_{1} = A_{0}C_{0}$, where: $A_{1}C_{1} = \sqrt{(A_{1}B_{1})^2 + (B_{1}C_{1})^2}$. As a result, it is shown that each shape is shrunk by an dilation factor of $\frac{1}{\sqrt{5}}$, for $\sqrt{5} \times \frac{1}{\sqrt{5}} = 1$:
\begin{equation}
    \begin{aligned}
        A_{i-1}B_{i-1} &= \frac{A_{i}B_{i}}{\sqrt{5}} \\
        B_{i-1}C_{i-1} &= \frac{B_{i}C_{i}}{\sqrt{5}} \\
        A_{i-1}C_{i-1} &= \frac{A_{i}C_{i}}{\sqrt{5}}
    \end{aligned}
\end{equation}

\noindent
Likewise, the area of each subsequent triangle $\triangle A_{i}B_{i}C_{i}$ is shrunk by the same factor. This can be proven by first finding the area of the subtriangle by simple division:

\begin{equation}
    \begin{aligned}
        A_{0}B_{0} &= x \\
        B_{0}C_{0} &= 2x \\
        \big<\triangle A_{0}B_{0}C_{0}\big> &= \frac{A_{0}B_{0} \times B_{0}C_{0}}{2} \\
        \big<\triangle A_{0}B_{0}C_{0}\big> &= x^2 \\
        \frac{\big<\triangle A_{1}B_{1}C_{1}\big>}{5} &= \frac{\big<\triangle A_{0}B_{0}C_{0}\big>}{5} \\
        \frac{\big<\triangle A_{1}B_{1}C_{1}\big>}{5} &= \frac{x^2}{5}
    \end{aligned}
\end{equation}

\noindent
Followed by calculating the area of the subtriangle via $A = \frac{l\times h}{2}$. Both values are equal:

\begin{equation}
    \begin{aligned}
        A_{1}B_{1} &= \frac{A_{0}B_{0}}{\sqrt{5}} = \frac{x}{\sqrt{5}}\\
        B_{1}C_{1} &= \frac{B_{0}C_{0}}{\sqrt{5}} = \frac{2x}{\sqrt{5}}\\
        \big<\triangle A_{1}B_{1}C_{1}\big> &= \frac{A_{1}B_{1} \times B_{1}C_{1}}{2} \\
        \big<\triangle A_{1}B_{1}C_{1}\big> &=  \frac{\frac{A_{0}B_{0}}{\sqrt{5}} \times \frac{B_{0}C_{0}}{\sqrt{5}}}{2} \\
        \big<\triangle A_{1}B_{1}C_{1}\big> &=  \frac{\frac{x}{\sqrt{5}} \times \frac{2x}{\sqrt{5}}}{2} \\
        \big<\triangle A_{1}B_{1}C_{1}\big> &=  \frac{x^2}{5}
    \end{aligned}
\end{equation}

\subsection{Geometric sequence of the area}
Recall how the area of each triangle decreases by an dilation factor of $\frac{1}{\sqrt{5}}$:
\begin{equation}
    \begin{aligned}
        \big<\triangle A_{0}B_{0}C_{0}\big> &= x^2,\\
        \big<\triangle A_{1}B_{1}C_{1}\big> &= \frac{x^2}{5},\\
        \big<\triangle A_{2}B_{2}C_{2}\big> &= \frac{x^2}{25},\\
        \big<\triangle A_{3}B_{3}C_{3}\big> &= \frac{x^2}{125},\\
        \ldots
    \end{aligned}
\end{equation}

\noindent
It is apparent that the decreasing area of the pinwheel triangle follow the following geometric sequence:
\begin{equation}
    x^2, \frac{x^2}{5}, \frac{x^2}{25}, \frac{x^2}{125}, \frac{x^2}{625}, \ldots
\end{equation}
\begin{equation}
    x^2, x^2\left(\frac{1}{5}\right)^1, x^2\left(\frac{1}{5}\right)^2, x^2\left(\frac{1}{5}\right)^3, x^2\left(\frac{1}{5}\right)^4, \ldots \\
\end{equation}

\noindent
Which generalizes to the form $\varepsilon + \varepsilon r^1 + \varepsilon r^2 + \ldots$, with $\varepsilon = x^2, r = \frac{1}{5}$. The notation $\varepsilon$ is used in this case, as it stands for the word area in Greek (\emph{έκταση})

\newpage
\section{Properties of the pinwheel spiral}
By connecting each vertice in $\triangle A_{0}B_{0}C_{0}$ with it's corresponding vertice-pair in subsequent subtriangles, a \emph{`vertex spiral'} can be constructed:

\begin{enumerate}
    \item Draw line $A_{0}A_{1}$, $A_{1}A_{2}$, $A_{2}A_{3}$ \ldots\ for all $A_{i}A_{i + 1}$
    \item Draw line $B_{0}B_{1}$, $B_{1}B_{2}$, $B_{2}B_{3}$ \ldots\ for all $B_{i}B_{i + 1}$
    \item Draw line $C_{0}C_{1}$, $C_{1}C_{2}$, $C_{2}C_{3}$ \ldots\ for all $C_{i}C_{i + 1}$
\end{enumerate}

\begin{figure}[H]
    \centering
    \begin{tikzpicture}[scale=1, rotate=90]
        \coordinate [label=above: $A_{0}$] (A_0) at (0,10);
        \coordinate [label=below: $B_{0}$] (B_0) at (0,0);
        \coordinate [label=right: $C_{0}$] (C_0) at (20,0);
        \draw[dotted] (A_0) -- (B_0) -- (C_0) -- (A_0);

        \coordinate [label=above: $A_{1}$] (A_1) at ($(A_0)!(B_0)!(C_0)$);
        \coordinate [label=below: $C_{1}$] (C_1) at ($(B_0)!0.5!(C_0)$);
        \coordinate (H_0) at ($(A_0)!(C_1)!(C_0)$);
        \coordinate [label=left: $B_{1}$] (B_1) at ($(A_1)!(C_1)!(B_0)$);
        \coordinate [label=above: $A_{2}$] (A_2) at ($(A_1)!(B_1)!(C_1)$);
        \coordinate [label=below: $C_{2}$] (C_2) at ($(B_1)!0.5!(C_1)$);
        \coordinate (H_1) at ($(A_1)!(C_2)!(C_1)$);
        \coordinate [label=left: $B_{2}$] (B_2) at ($(A_2)!(C_2)!(B_1)$);

        \coordinate (A_3) at ($(A_2)!(B_2)!(C_2)$);
        \coordinate (C_3) at ($(B_2)! 0.5 !(C_2)$);
        \coordinate (H_2) at ($(A_2)!(C_3)!(C_2)$);
        \coordinate (B_3) at ($(A_3)!(C_3)!(B_2)$);
        \coordinate (A_4) at ($(A_3)!(B_3)!(C_3)$);
        \coordinate (C_4) at ($(B_3)! 0.5 !(C_3)$);
        \coordinate (H_3) at ($(A_3)!(C_4)!(C_3)$);
        \coordinate (B_4) at ($(A_4)!(C_4)!(B_3)$);
        \coordinate (A_5) at ($(A_4)!(B_4)!(C_4)$);
        \coordinate (C_5) at ($(B_4)! 0.5 !(C_4)$);
        \coordinate (H_4) at ($(A_4)!(C_5)!(C_4)$);
        \coordinate (B_5) at ($(A_5)!(C_5)!(B_4)$);
        \coordinate (A_6) at ($(A_5)!(B_5)!(C_5)$);
        \coordinate (C_6) at ($(B_5)! 0.5 !(C_5)$);
        \coordinate (H_5) at ($(A_5)!(C_6)!(C_5)$);
        \coordinate (B_6) at ($(A_6)!(C_6)!(B_5)$);
        \coordinate (A_7) at ($(A_6)!(B_6)!(C_6)$);
        \coordinate (C_7) at ($(B_6)! 0.5 !(C_6)$);
        \coordinate (H_6) at ($(A_6)!(C_7)!(C_6)$);
        \coordinate (B_7) at ($(A_7)!(C_7)!(B_6)$);
        \coordinate (A_8) at ($(A_7)!(B_7)!(C_7)$);
        \coordinate (C_8) at ($(B_7)! 0.5 !(C_7)$);
        \coordinate (H_7) at ($(A_7)!(C_8)!(C_7)$);
        \coordinate (B_8) at ($(A_8)!(C_8)!(B_7)$);
        \coordinate (A_9) at ($(A_8)!(B_8)!(C_8)$);
        \coordinate (C_9) at ($(B_8)! 0.5 !(C_8)$);
        \coordinate (H_8) at ($(A_8)!(C_9)!(C_8)$);
        \coordinate (B_9) at ($(A_9)!(C_9)!(B_8)$);

        \draw (A_0) -- (A_1) -- (A_2) -- (A_3) -- (A_4) -- (A_5) -- (A_6) -- (A_7) -- (A_8) -- (A_9);

        \draw (B_0) -- (B_1) -- (B_2) -- (B_3) -- (B_4) -- (B_5) -- (B_6) -- (B_7) -- (B_8) -- (B_9);

        \draw (C_0) -- (C_1) -- (C_2) -- (C_3) -- (C_4) -- (C_5) -- (C_6) -- (C_7) -- (C_8) -- (C_9);

    \end{tikzpicture}
    \label{pinwheel-spiral}
\end{figure}


\noindent
Note how there are 3 spirals, depending on the vertice used in it's construction. Let the 3 spirals be defined as the $A_{i}$ spiral, $B_{i}$ spiral, and the $C_{i}$ spiral, hereby to be referred as \emph{vertex spirals}:

\begin{figure}[H]
    \centering
    \begin{tikzpicture}[rotate=0]
      \begin{scope}[scale=8]
        \coordinate [label=left: $A_{0}$] (A-0) at (0,1);
        \coordinate (B-0) at (0,0);
        \coordinate (C-0) at (2,0);
      \end{scope}
      %\draw[dotted, thin] (A-0) -- (B-0) -- (C-0) -- (A-0);

      \foreach \lev in {1,...,20}{
        \pgfmathsetmacro{\plev}{int(\lev-1)}
        \coordinate (A-\lev) at ($(A-\plev)!(B-\plev)!(C-\plev)$);
        \coordinate (C-\lev) at ($(B-\plev)!.5!(C-\plev)$);
        \coordinate (H-\plev) at ($(A-\plev)!(C-\lev)!(C-\plev)$);
        \coordinate (B-\lev) at ($(A-\lev)!(C-\lev)!(B-\plev)$);

        \draw[thin] (A-\lev) -- (A-\plev);
      }
    \end{tikzpicture}
    \label{pinwheel-triangle-infinite-large}
    \caption{Detail view of the \protect$A_n$ spiral}
\end{figure}


\begin{figure}[H]
    \centering
    \begin{tikzpicture}[rotate=0]
      \begin{scope}[scale=8]
        \coordinate (A-0) at (0,1);
        \coordinate [label=right: $B_{0}$] (B-0) at (0,0);
        \coordinate (C-0) at (2,0);
      \end{scope}
      %\draw[dotted, thin] (A-0) -- (B-0) -- (C-0) -- (A-0);

      \foreach \lev in {1,...,20}{
        \pgfmathsetmacro{\plev}{int(\lev-1)}
        \coordinate (A-\lev) at ($(A-\plev)!(B-\plev)!(C-\plev)$);
        \coordinate (C-\lev) at ($(B-\plev)!.5!(C-\plev)$);
        \coordinate (H-\plev) at ($(A-\plev)!(C-\lev)!(C-\plev)$);
        \coordinate (B-\lev) at ($(A-\lev)!(C-\lev)!(B-\plev)$);

        \draw[thin] (B-\lev) -- (B-\plev);
      }
    \end{tikzpicture}
    \label{pinwheel-triangle-infinite-large}
    \caption{Detail view of the \protect$B_n$ spiral}
\end{figure}


\begin{figure}[H]
    \centering
    \begin{tikzpicture}[rotate=0]
      \begin{scope}[scale=8]
        \coordinate (A-0) at (0,1);
        \coordinate (B-0) at (0,0);
        \coordinate [label=above: $C_{0}$] (C-0) at (2,0);
      \end{scope}
      %\draw[dotted, thin] (A-0) -- (B-0) -- (C-0) -- (A-0);

      \foreach \lev in {1,...,20}{
        \pgfmathsetmacro{\plev}{int(\lev-1)}
        \coordinate (A-\lev) at ($(A-\plev)!(B-\plev)!(C-\plev)$);
        \coordinate (C-\lev) at ($(B-\plev)!.5!(C-\plev)$);
        \coordinate (H-\plev) at ($(A-\plev)!(C-\lev)!(C-\plev)$);
        \coordinate (B-\lev) at ($(A-\lev)!(C-\lev)!(B-\plev)$);

        \draw[thin] (C-\lev) -- (C-\plev);
      }
    \end{tikzpicture}
    \label{pinwheel-triangle-infinite-large}
    \caption{Detail view of the \protect$C_n$ spiral}
\end{figure}


\newpage
\subsection{Geometric series and the length of vertex spirals}
In order for us to find the length of each vertex spiral (hereby denoted as $\mathbb{A}, \mathbb{B}, \mathbb{C}$), we would have to first find the individual length of each spiral segment (e.g. $A_{0}A_{1}, A_{1}A_{2}, A_{2}A_{3}$). It is trivial to state that the length of $\mathbb{A}, \mathbb{B}, \mathbb{C}$ is the sum of the length of it's dependent segments:
\begin{equation}
    \begin{aligned}
        \mathbb{A} &= \sum_{i = 0}^{\infty} A_{i}A_{i+1} \\
        \mathbb{B} &= \sum_{i = 0}^{\infty} B_{i}B_{i+1} \\
        \mathbb{C} &= \sum_{i = 0}^{\infty} C_{i}C_{i+1} \\
    \end{aligned}
\end{equation}

\noindent
Now recall that the dilation factor of each subsequent pinwheel triangle equals $\frac{1}{\sqrt{5}}$. Therefore, the length of each subsequent vertex spiral segment follows a simple geometric series, where for example:

\begin{equation}
    \begin{aligned}
        A_{0}A_{1} &= x \times \frac{1}{\sqrt{5}} \\
        A_{1}A_{2} &= x \times \big(\frac{1}{\sqrt{5}}\big)^2 \\
        A_{2}A_{3} &= x \times \big(\frac{1}{\sqrt{5}}\big)^3 \\
        A_{3}A_{4} &= x \times \big(\frac{1}{\sqrt{5}}\big)^4 \\
        \ldots
    \end{aligned}
\end{equation}

\noindent
Which is of the general form $a + ar^1 + ar^2 + \ldots$, with $a = x, r = \frac{1}{\sqrt(5)}$. The same holds true for segments of the order $B_{i}$ and $C_{i}$:

\begin{equation}
    \begin{aligned}
        a + ar^1 + ar^2 + \ldots, a &= x, &r = \frac{1}{\sqrt{5}} \\
        b + br^1 + br^2 + \ldots, b &= 2x, &r = \frac{1}{\sqrt{5}} \\
        c + cr^1 + cr^2 + \ldots, c &= \sqrt{(x)^2 + (2x)^2}, &r = \frac{1}{\sqrt{5}} \\
    \end{aligned}
\end{equation}

\subsection{Approximating the absolute length of vertex spirals through graphical methods}
A simple way to approximate the absolute length (denoted as $\mathbb{A}, \mathbb{B}, \mathbb{C}$) of the vertex spirals is to graph their accumulative length on a cartesian plane:

\setlength{\parindent}{0ex}
\begin{tikzpicture}[trim axis left]
    \begin{axis}[
            scale only axis,
            title=Accumulative length of vertex segments,
            xlabel={Number of segments},
            ylabel={Accumulative length in factors of $x$},
            grid=major,
            ymin=1,
            height=7cm,
            width=\textwidth,
            xmin=0, xmax=10,
            legend entries={$\mathbb{A}$,$\mathbb{B}$,$\mathbb{C}$},
        ]
        \addplot [blue] table {figures/a-spiral.dat};
        \addplot [red] table {figures/b-spiral.dat};
        \addplot [green] table {figures/c-spiral.dat};
    \end{axis}
\end{tikzpicture}


\subsection{Finding the absolute length of vertex spirals through the use of limits}
Now we can visually see that the length of each vertex segment rapidly convergences upon a certain value:
\begin{equation}
    \begin{aligned}
        \mathbb{A} &\approx 1.81x \\
        \mathbb{B} &\approx 3.62x \\
        \mathbb{C} &\approx 4.04x \\
    \end{aligned}
\end{equation}

\noindent
However, what could be done in order to find the absolute value of each vertex segment's length? As we have already established, the length of each vertex segment can be computed through the summation of an geometric sequence:

\begin{equation}
    \begin{aligned}
        \sum_{i=1}^{\infty}a_i = \left(\frac{1-r^n}{1-r}\right)
    \end{aligned}
\end{equation}

Such a sequence is infinite, therefore by summing up this \emph{infinite geometric sequence}, it is possible for us to reach the absolute value. In order to do this, we will utilize the special form of the formula that calculates the first $n$th terms of a geometric series, where the ratio geometric ratio $|r| < 1$. Since $r = \frac{1}{\sqrt{5}}$, this condition holds true:

\begin{equation}
    \begin{aligned}
        \sum_{i=1}^{\infty}a_i=\frac{a}{1-r}
    \end{aligned}
\end{equation}

\noindent
By applying the above formula, we obtain the following equations for the length of vertex spiral $\mathbb{A}$, $\mathbb{B}$, and $\mathbb{C}$:

\begin{equation}
    \begin{aligned}
        \mathbb{A} = \lim_{n \to \infty} \sum_{i=0}^{n} A_0\left(\frac{1}{\sqrt{5}}\right)^i &=\frac{A_0}{1-\frac{1}{\sqrt{5}}} \\
        \mathbb{B} = \lim_{n \to \infty} \sum_{i=0}^{n} B_0\left(\frac{1}{\sqrt{5}}\right)^i &=\frac{B_0}{1-\frac{1}{\sqrt{5}}} \\
        \mathbb{C} = \lim_{n \to \infty} \sum_{i=0}^{n} C_0\left(\frac{1}{\sqrt{5}}\right)^i &=\frac{C_0}{1-\frac{1}{\sqrt{5}}} \\
    \end{aligned}
\end{equation}

\noindent
Therefore, we can obtain the absolute value of the length of these vertex spiral segments, by substituting $A_0$, $B_0$, and $C_0$ with their respective values:

\begin{equation}
    \begin{aligned}
        \mathbb{A} &=\frac{x}{1-\frac{1}{\sqrt{5}}} \\
        \mathbb{B} &=\frac{2x}{1-\frac{1}{\sqrt{5}}} \\
        \mathbb{C} &=\frac{\sqrt{(x)^2 + (2x)^2}}{1-\frac{1}{\sqrt{5}}}
    \end{aligned}
\end{equation}

\noindent
By solving for factors of x, this results in numerical solutions that are in alignment with our graphical approximation from the previous section:
\begin{equation}
    \begin{aligned}
        \mathbb{A} &=\frac{x}{1-\frac{1}{\sqrt{5}}} &\approx 1.809016994x\\
        \mathbb{B} &=\frac{2x}{1-\frac{1}{\sqrt{5}}} &\approx 3.618033989x\\
        \mathbb{C} &=\frac{\sqrt{(x)^2 + (2x)^2}}{1-\frac{1}{\sqrt{5}}} &\approx 4.045084972x
    \end{aligned}
\end{equation}
