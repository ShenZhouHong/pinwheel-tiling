\section{Conclusion}
In conclusion, we have explored many interesting properties of the pinwheel triangle, and the pinwheel spiral. We were able to first touch on the recursive and convergent nature of the pinwheel triangle by noting it's area in the form of a geometric sequence, bringing order to an otherwise unorderly aperiodic tiling.

Using that as a foundation, we then constructed a pinwheel spiral, where we are able to calculate the length of each individual segment using it's dilation factor, as well as find a means of approximating the total length of each spiral segment through a graphical approach. From this graphical approach, we were able to see that an limit exists, upon which the values converge. This allowed us to formally use limits and the summation of an infinite geometric sequence in order to find the absolute value of each vertex segment's length.

Overall, this investigation allowed us to explore the properties of the pinwheel triangle and a pinwheel spiral, in ways that no mathematican has explored previously. It allowed me to appreciate the beauty of these recursive constructs, as well as the subtle semblance of order that exists in something which looks disorderly.

\section{Evaluation}
I believe a major weak point of my study was being unable to find the exact cartesian coordinate where the spiral converges upon. Such an task would require the study of matrixes, which is beyond my ability. However, a lesser task of proving the existence of an convergence formally would be possible, but alas - I was unable to approach it due to the lack of time.

Likewise, another avenue for exploration is a generalization of these principles to other triangles as well, and other shapes of aperiodic and recursive construction.

\subsection{Further fields of study}
\begin{enumerate}
    \item A formal prove for the existence of an convergence point in cartesian coordinate space of the pinwheel spiral
    \item Finding the cartesian coordinates of such an convergence point
\end{enumerate}
